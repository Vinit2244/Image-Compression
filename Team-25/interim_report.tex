\documentclass{article}
\usepackage{graphicx}
\usepackage[margin=4cm]{geometry}

\begin{document}

\title{
    {Interim Report: Linear Algebra Project}\\
    {\large International Institute of Information Technology, Hyderabad}\\
    \author{Vinit Mehta, Swarang Joshi, Nijesh Raghava}
    \vspace{1cm}
    \includegraphics[width=3cm]{IIITH.png}
}
\date{}

\maketitle

\section{Introduction}
This interim report provides an update on our ongoing linear algebra project on image compression techniques. The project aims to explore and compare different methods of image compression, specifically focusing on vector quantization, principal component analysis (PCA), discrete cosine transform (DCT), and wavelet transform. The report outlines the problem statement, related works, use case of linear algebra in the project, tentative timeline plan, and individual work contributions.

\section{Related Works}
To better understand the current state of image compression techniques, we conducted a literature review and analyzed several research papers. Some notable works in this field include:
\begin{itemize}
    \item Hu, Y., Chen, W., Lo, C. and Chuang, J.. "Improved vector quantization scheme for grayscale image compression" Opto-Electronics Review, vol. 20, no. 2, 2012, pp. 187-193. https://doi.org/10.2478/s11772-012-0016-z
    \item Jolliffe Ian T. and Cadima Jorge 2016Principal component analysis: a review and recent developmentsPhil. Trans. R. Soc. A.3742015020220150202\\
    http://doi.org/10.1098/rsta.2015.0202
    \item Khan, Sulaiman, et al. "An efficient JPEG image compression based on Haar wavelet transform, discrete cosine transform, and run length encoding techniques for advanced manufacturing processes." Measurement and Control 52.9-10 (2019): 1532-1544.
    \item Starosolski, Roman. "Hybrid adaptive lossless image compression based on discrete wavelet transform." Entropy 22.7 (2020): 751.
    \item Renkjumnong, Wasuta -., "SVD and PCA in Image Processing." Thesis, Georgia State University, 2007.
    doi: https://doi.org/10.57709/1059687 
\end{itemize}
These papers provided valuable insights into the various techniques and algorithms employed in image compression, specifically those that utilize linear algebra.

\section{Problem Statement/ Formulation}
The problem statement of our project is to investigate and compare different image compression techniques that utilize linear algebra, specifically focusing on vector quantization, principal component analysis, discrete cosine transform, and wavelet transform. We aim to understand the underlying mathematical principles, advantages, and limitations of each technique. Furthermore, we plan to implement these techniques and evaluate their performance on a common example.

\section{Use Case of Linear Algebra in the Project}
Linear algebra plays a crucial role in image compression techniques. It provides the mathematical foundation for representing and manipulating images as matrices and vectors. Specifically, vector quantization relies on linear algebra to quantize image vectors into codebooks, while principal component analysis utilizes linear algebra to find orthogonal basis vectors that capture the most significant variations in the image data. Similarly, discrete cosine transform and wavelet transform employ linear algebra to transform the image into a compressed representation by exploiting certain properties of the image signals.

\section{Tentative Timeline Plan}
Our tentative timeline plan for the remainder of the project is as follows:

\begin{itemize}
    \item \textbf{Day 1 and 2:} Conduct literature review on image compression techniques, focusing on vector quantization, principal component analysis, discrete cosine transform, and wavelet transform. Study and analyze the selected research papers, understanding the mathematical principles and algorithms behind each technique.
    \item \textbf{Day 3:} Implement vector quantization, principal component analysis, discrete cosine transform, and wavelet transform algorithms.
    \item \textbf{Day 4:} Prepare a common example image and compress it using all three techniques.
    \item \textbf{Day 5 and 6:} Analyze the results and draw conclusions on the effectiveness and suitability of each technique. Begin finalizing the project report and presentation.
\end{itemize}

\section{Individual Work Contribution}
Each team member will do the following works:

\begin{itemize}
    \item Vinit Mehta: Prepared Interim Report in Latex. Will conduct the literature review, analyse research papers, and contribute to the implementation of the image compression techniques. Exploring the application of principal component analysis and implementing the principal component analysis algorithm.
    \item Swarang Joshi: Will study and analyse the mathematical principles and algorithms of the selected techniques, implement the vector quantization algorithm, and assist in the evaluation and analysis of the results.
    \item Nijesh Raghava: Will explore the application of discrete cosine transform, and wavelet transform, and assist in the evaluation and analysis of the results. Implement the wavelet transform algorithm and assist in the evaluation and analysis of the results.
\end{itemize}

\end{document}
